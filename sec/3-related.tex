% 1-2page Related work goes here

\section{Related Work}
\label{sec:related}

\cch[Explain the reason for doing this work, why is this problem important]

	% \begin{subsection}{ROI Extraction}
	% Fingerprint recognition has been widely studied, and various approaches have been proposed for ROI extraction.
	% Processing the entire fingerprint image is a computationally expensive task.
	% Extracting the ROI from the fingerprint image before processing it further is necessary.
	% A brief overview of some of the related works is presented in this section.
	%% using the entire fingerprint
	% Segmentation of fingerprint images into discreet blocks and then clustering based on a specific threshold value is a widely used approach \cite{wahhab2019novel,andrezza2021simple}.
	% One such work evaluated the area of interest based on a quality metric called "coherence" \cite{wahhab2019novel}, an experimental threshold value, 
	% to decide whether the block is in the foreground. 
	% The fixed block size and the specified threshold value are the limitations of the work.
	% %
	% Various approaches have been proposed in the past for extracting ROI (Region of Interest) from fingerprint images.
	% These approaches can be broadly classified into the following categories:
	% \begin{itemize}
	% 	\item \textbf{Segmentation-Based Approach:} This approach involves using techniques such as watershed segmentation \cite{sumijan2020fingerprint}, or active contour models \cite{hilles2021adaptive} to segment the fingerprint image and extract the ROI. However, this approach may suffer from over-segmentation or under-segmentation, which can affect the accuracy of ROI extraction.
	% 	\item \textbf{Filtering-Based Approach:} This approach involves using filters such as Sobel, Prewitt, or Canny \cite{sujatha2015performance}, to detect the edges in the fingerprint image, followed by thresholding \cite{lupu2014development} to extract the ROI. However, this approach may be affected by noise and variations in fingerprint patterns, which can limit its accuracy.
	% 	\item \textbf{Clustering-Based Approach:} This approach involves using clustering techniques such as K-means clustering \cite{mehdi2019improving} to group similar pixels in the fingerprint image and extract the ROI. However, the accuracy of clustering may be affected by the choice of initial centroids and the number of clusters.
	% 	\item \textbf{Hybrid Approaches:} These approaches combine the advantages of different methods, such as filtering and segmentation \cite{zhang2013adaptive}, to improve the accuracy and efficiency of ROI extraction. The proposed approach combines segmentation with filtering using statistical measures and agglomerative clustering to extract the ROI from fingerprint images.
	% 	\item \textbf{Deep Learning-Based Approaches:} In recent years, deep learning-based approaches \cite{joshi2021sensor,wan2020xfinger}, such as convolutional neural networks (CNNs) \cite{li2018fingerprint}, have been used for ROI extraction from fingerprint images. These approaches can achieve high accuracy but require large training data and computing resources.
	% \end{itemize}
	% In summary, there have been various approaches proposed in the past for ROI extraction from fingerprint images, each with its advantages and limitations. 
	% The proposed clustering-based hybrid approach aims to overcome the limitations of existing methods and provide a more accurate and efficient solution for ROI extraction.
	% The approach is computationally inexpensive and can be used for real-time applications while being able to handle noise and variations in fingerprint patterns.
	% This allows for generalizing the approach to a wide range of fingerprint images obtained from different sensors with different noise levels and variations in fingerprint patterns.
	%% Most basic approach
	% One popular approach for ROI extraction is based on filters. 
	% A filter is a mathematical operation that is applied to an image to enhance or extract certain features. In this case, the filter is used to extract the edges of the fingerprint image.
	% Sobel and Prewitt filters \cite{sujatha2015performance}, the most basic edge detection filters, have been used to detect edges in fingerprint images, 
	% and the resulting edge map is used to extract the ROI. 
	% However, the accuracy of edge detection can be affected by noise and variations in fingerprint patterns, which limits the performance of this approach.
	% To attenuate the effect of noise, work has been done to improve the performance of the Sobel filter by using a median filter \cite{zhang2013adaptive}.
	% The median filter is used to remove noise from the image, and the Sobel filter is used to extract the edges of the fingerprint image.
	% The resulting edge map is used to extract the ROI. The performance of this approach is better than that of the Sobel filter alone, but it is still prone to noise and variations in fingerprint patterns.
	% An approach for optimal filter obtained through convolution method, which was used for image enhancement and ROI extraction, was able to extract the ROI with high accuracy  with the only drawback being that the approach was computationally expensive.
	% \end{subsection}

\cch[How was it done in the very beginning]
	\cc[Try to cite \cite{peddi2023ROIClustering} here]
\cch[How much work has been done since then]
	% Standard deviation, median and binarization filter to identify area of interest using parameters like median window size, binarization threshold and the sizes
	% of the inner block and outer block also have been used \cite{andrezza2021simple}. These parameters had experimental threshold values, which were not optimal for all the images.

\cch[What are the research Gaps]

\cch[How this work is different from the existing work]
	% The proposed approach uses a combination of statistical measures with simple clustering to extract the ROI from the fingerprint image.

\cch[Background for the proposed work]
% A brief description of the background/preliminaries of the work. 
% Briefly to the point and precisely (1-2 pages).








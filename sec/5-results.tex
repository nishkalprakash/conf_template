% 1 page results
\section{Experimental Results}
\label{sec:results}

\{Detail design of experiment/setup/protocols/subjects/data collection/ or a detailed description of the used datasets\}

\{Objectives of the experiment/what do you want to prove/substantiate?\}

\{Experimental Environment, with a link to code/data repository\}

\{Result vis-a-vis experimental objectives with proper tables, graphs, plots, etc\}


	% \begin{comment}
	
	% \setlength{\belowdisplayskip}{5pt} \setlength{\belowdisplayshortskip}{2pt}
	% \setlength{\abovedisplayskip}{2pt} \setlength{\abovedisplayshortskip}{0pt}
	% For this experiment, two datasets, FVC2002 DB2 and SPD2010, have been used to check the region consistency and prove the core point's existence in the ROI, respectively.For the block size, statistical metrics, and clustering methods both datasets are used combinedly.

	% % \subsubsection{Region Consistency:}
	% \textbf{\textit{Region Consistency:}}
	% To prove that the region obtained is consistent across all impressions of the same fingerprint, a fingerprint is considered from the FVC dataset. The ROI computed for these fingerprints is shown in figure \ref{fig:roi_fvc}, and it can be seen that the ROI is consistent.

	% \begin{figure}[H]
	% 	\begin{subfigure}[b]{0.12\textwidth}
	% 		\centering
	% 		\includegraphics[width=\textwidth]{images/fvc12.png}
	% 		% \caption{ }
	% 		\label{conf_1}
	% 	\end{subfigure}
	% 	\hfill
	% 	\begin{subfigure}[b]{0.12\textwidth}
	% 		\centering
	% 		\includegraphics[width=\textwidth]{images/fvc13.png}
	% 		% \caption{ }
	% 		\label{conf_2}
	% 	\end{subfigure}
	% 	\hfill
	% 	\begin{subfigure}[b]{0.12\textwidth}
	% 		\centering
	% 		\includegraphics[width=\textwidth]{images/fvc14.png}
	% 		% \caption{ }
	% 		\label{conf_3}
	% 	\end{subfigure}
	% 	\hfill
	% 	\begin{subfigure}[b]{0.12\textwidth}
	% 		\centering
	% 		\includegraphics[width=\textwidth]{images/fvc15.png}
	% 		% \caption{ }
	% 		\label{conf_3}
	% 	\end{subfigure}
	% 	\hfill
	% 	\begin{subfigure}[b]{0.12\textwidth}
	% 		\centering
	% 		\includegraphics[width=\textwidth]{images/fvc16.png}
	% 		% \caption{ }
	% 		\label{conf_3}
	% 	\end{subfigure}
	% 	\hfill
	% 	\begin{subfigure}[b]{0.12\textwidth}
	% 		\centering
	% 		\includegraphics[width=\textwidth]{images/fvc_11.png}
	% 		% \caption{ }
	% 		\label{conf_3}
	% 	\end{subfigure}
	% 	% \begin{subfigure}[b]{0.12\textwidth}
	% 	%      \centering
	% 	%      \includegraphics[width=\textwidth]{images/syn_11.png}
	% 	%      \caption{ }
	% 	%      \label{conf_1}
	% 	%  \end{subfigure}
	% 	%  \hfill
	% 	%  \begin{subfigure}[b]{0.12\textwidth}
	% 	%      \centering
	% 	%      \includegraphics[width=\textwidth]{images/syn_12.png}
	% 	%      \caption{ }
	% 	%      \label{conf_2}
	% 	%  \end{subfigure}
	% 	%  \hfill
	% 	% \begin{subfigure}[b]{0.12\textwidth}
	% 	%      \centering
	% 	%      \includegraphics[width=\textwidth]{images/syn_13.png}
	% 	%      \caption{ }
	% 	%      \label{conf_3}
	% 	%  \end{subfigure}
	% 	%  \hfill
	% 	%  \begin{subfigure}[b]{0.12\textwidth}
	% 	%      \centering
	% 	%      \includegraphics[width=\textwidth]{images/syn_14.png}
	% 	%      \caption{ }
	% 	%      \label{conf_3}
	% 	%  \end{subfigure}
	% 	%  \hfill
	% 	%  \begin{subfigure}[b]{0.12\textwidth}
	% 	%      \centering
	% 	%      \includegraphics[width=\textwidth]{images/syn_15.png}
	% 	%      \caption{ }
	% 	%      \label{conf_3}
	% 	%  \end{subfigure}
	% 	% \hfill
	% 	%  \begin{subfigure}[b]{0.12\textwidth}
	% 	%      \centering
	% 	%      \includegraphics[width=\textwidth]{images/syn_16.png}
	% 	%      \caption{ }
	% 	%      \label{conf_3}
	% 	%  \end{subfigure}

	% 	\caption{ROI obtained for different impressions of a fingerprint.}
	% 	\label{fig:roi_fvc}
	% \end{figure}

	% \textit{\textbf{ROI with Corepoint:}}
	% As the SPD dataset contains the core point labels, it has been used to show that the ROI always contains a core point. Six fingerprints are selected at random, and their core points and ROI are shown in figure \ref{fig:core_spd}.

	% \begin{figure}[H]
	% 	\begin{subfigure}[b]{0.12\textwidth}
	% 		\centering
	% 		\includegraphics[width=\textwidth]{images/spd1.png}
	% 		% \caption{ }
	% 		\label{conf_1}
	% 	\end{subfigure}
	% 	\hfill
	% 	\begin{subfigure}[b]{0.12\textwidth}
	% 		\centering
	% 		\includegraphics[width=\textwidth]{images/spd2.png}
	% 		% \caption{ }
	% 		\label{conf_2}
	% 	\end{subfigure}
	% 	\hfill
	% 	\begin{subfigure}[b]{0.12\textwidth}
	% 		\centering
	% 		\includegraphics[width=\textwidth]{images/spd3.png}
	% 		% \caption{ }
	% 		\label{conf_3}
	% 	\end{subfigure}
	% 	\hfill
	% 	\begin{subfigure}[b]{0.12\textwidth}
	% 		\centering
	% 		\includegraphics[width=\textwidth]{images/spd4.png}
	% 		% \caption{ }
	% 		\label{conf_3}
	% 	\end{subfigure}
	% 	\hfill
	% 	\begin{subfigure}[b]{0.12\textwidth}
	% 		\centering
	% 		\includegraphics[width=\textwidth]{images/spd5.png}
	% 		% \caption{ }
	% 		\label{conf_3}
	% 	\end{subfigure}
	% 	\hfill
	% 	\begin{subfigure}[b]{0.12\textwidth}
	% 		\centering
	% 		\includegraphics[width=\textwidth]{images/spd6.png}
	% 		% \caption{ }
	% 		\label{conf_3}
	% 	\end{subfigure}

	% 	%  \begin{subfigure}[b]{0.12\textwidth}
	% 	%      \centering
	% 	%      \includegraphics[width=\textwidth]{images/spd7.png}
	% 	%      \caption{ }
	% 	%      \label{conf_1}
	% 	%  \end{subfigure}
	% 	%  \hfill
	% 	%  \begin{subfigure}[b]{0.12\textwidth}
	% 	%      \centering
	% 	%      \includegraphics[width=\textwidth]{images/spd8.png}
	% 	%      \caption{ }
	% 	%      \label{conf_2}
	% 	%  \end{subfigure}
	% 	%  \hfill
	% 	% \begin{subfigure}[b]{0.12\textwidth}
	% 	%      \centering
	% 	%      \includegraphics[width=\textwidth]{images/spd9.png}
	% 	%      \caption{ }
	% 	%      \label{conf_3}
	% 	%  \end{subfigure}
	% 	%  \hfill
	% 	%  \begin{subfigure}[b]{0.12\textwidth}
	% 	%      \centering
	% 	%      \includegraphics[width=\textwidth]{images/spd10.png}
	% 	%      \caption{ }
	% 	%      \label{conf_3}
	% 	%  \end{subfigure}
	% 	%  \hfill
	% 	%  \begin{subfigure}[b]{0.12\textwidth}
	% 	%      \centering
	% 	%      \includegraphics[width=\textwidth]{images/spd11.png}
	% 	%      \caption{ }
	% 	%      \label{conf_3}
	% 	%  \end{subfigure}
	% 	% \hfill
	% 	%  \begin{subfigure}[b]{0.12\textwidth}
	% 	%      \centering
	% 	%      \includegraphics[width=\textwidth]{images/spd12.png}
	% 	%      \caption{ }
	% 	%      \label{conf_3}
	% 	%  \end{subfigure}
	% 	\caption{A set of images from the SPD dataset with their ROI and core point.}
	% 	\label{fig:core_spd}
	% \end{figure}
	% \textit{\textbf{Block Size:}}
	% Based on the observation of the ROI generated by the method, the corresponding block size is given a score of `1' if the ROI contains the core point, and a `0' otherwise. Thus, the scores for different block sizes are compared, and the block sizes within the range of 16 to 35 have a score of `1'. A block size of 23 is chosen for the method as it is close to the mid-point of the range of the optimal block sizes.

	% % \begin{figure}[h!]
	% %     \centering
	% %     \includegraphics[width=0.5\textwidth]{images/block_size.jpg}
	% %     \caption{Variation of the score with respect to block size.}
	% %     \label{fig:block_size}
	% % \end{figure}

	% %A block size of 23 is chosen for the method as it is close to the mid-point of the range of the optimal block sizes.

	% \textit{\textbf{Statistical Metrics:}} Different statistical metrics like mean, variance, median and mode are used to reshape features in section \ref{feature}. The features obtained from these statistical metrics are checked for their miss rate, and it can be seen from figure \ref{fig:miss_rate} that the mean of features has the least amount of miss. For this study, it is considered to be a miss whenever the ROI doesn't contain the core point.

	% \textit{\textbf{Clustering Methods:}} The features are clustered based on the k-means and agglomerative algorithms. The Silhouette scores for the k-means and agglomerative clustering algorithms are computed, and as can be seen in figure \ref{fig:clus_score}, agglomerative clustering has the best score for the three clusters.
	% \begin{figure}[H]
	% 	\begin{subfigure}[b]{0.45\textwidth}
	% 		\centering
	% 		\includegraphics[width=\textwidth]{images/chart(5).png}
	% 		% \caption{Miss rate calculated for SPD2010}
	% 		\caption{Miss Rate}
	% 		\label{fig:miss_rate}
	% 	\end{subfigure}
	% 	\hfill
	% 	\begin{subfigure}[b]{0.45\textwidth}
	% 		\centering
	% 		\includegraphics[width=\textwidth]{images/chart(6).png}
	% 		% \caption{Silhouette score for the clustering algorithms}
	% 		\caption{Silhouette Score}
	% 		\label{fig:clus_score}
	% 	\end{subfigure}
	% 	\hfill
	% 	\caption{Analysis of the Miss Rate and Silhouette Score.}
	% 	\label{fig:fr_perf}
	% \end{figure}
	% % \vspace{-1.5em}
	% \textit{\textbf{Fingerprint Recognition:}} The evaluation of fingerprint recognition accuracy involves using half of the data from each dataset as genuine data and the other half as imposter data. A random fingerprint is selected from either half of the data and compared with all fingerprints from the genuine half. Based on the Harris corner matching score \cite{derpanis2004harris}, the selected fingerprint is declared to either have a match or no match. The performance of the method for a set of fingerprints with and without ROI is shown in figure \ref{fig:fr_perf}.

	% \begin{figure}[H]
	% 	\begin{subfigure}[b]{0.5\textwidth}
	% 		\centering
	%    \includegraphics[width=\textwidth]{images/chart(7).png}
	% 		% \includegraphics[width=\textwidth]{images/Updated_1.png}
	% 		\caption{Without ROI}
	% 		\label{conf_1}
	% 	\end{subfigure}
	% 	\hfill
	% 	\begin{subfigure}[b]{0.45\textwidth}
	% 		\centering
	% 		% \includegraphics[width=\textwidth]{images/updated_2.png}
	%   \includegraphics[width=\textwidth]{images/chart(8).png}
	% 		\caption{With ROI}
	% 		\label{conf_2}
	% 	\end{subfigure}
	% 	\hfill
	% 	\caption{Performance of the fingerprint recognition method}
	% 	\label{fig:fr_perf}
	% \end{figure}
	% Additionally, the performance of the fingerprint recognition method in terms of FAR, FRR, GAR and GRR is tabulated in table \ref{tab:fr_perf}. It can be seen that the performance of the method is better when the fingerprint is used along with an ROI.

	% \textit{\textbf{Comaprison of Time and Accuracy:}}
	% The average identification time is compared between the full fingerprint and the ROI in the dataset, as shown in Figure \ref{time_Comparison}. Additionally, Figure \ref{Acc_Comp} provides a comparison of the proposed method's accuracy with existing Cluster-based \cite{mehdi2019improving} and Filter-based \cite{841531} methods.

	% \begin{figure}[H]
	% 	\begin{subfigure}[b]{0.45\textwidth}
	% 		\centering
	%    \includegraphics[width=\textwidth]{images/premi_time_Comparison.png}
	% 		% \includegraphics[width=\textwidth]{images/Updated_1.png}
	% 		\caption{Time Comparison}
	% 		\label{time_Comparison}
	% 	\end{subfigure}
	% 	\hfill
	% 	\begin{subfigure}[b]{0.45\textwidth}
	% 		\centering
	% 		% \includegraphics[width=\textwidth]{images/updated_2.png}
	%   \includegraphics[width=\textwidth]{images/Comparison(1).png}
	% 		\caption{Accuracy comparison}
	% 		\label{Acc_Comp}
	% 	\end{subfigure}
	% 	\hfill
	% 	\caption{Time comparison and Accuracy comparison}
	% 	\label{fig:fr_perf}
	% \end{figure}

	% % \begin{table}[h]
	% % \setlength{\arrayrulewidth}{0.1mm}
	% % \setlength{\tabcolsep}{5pt}
	% % \renewcommand{\arraystretch}{1.5}
	% % \centering
	% % \caption{Fingerprint Recognition Performance with ROI}
	% % \label{tab:fr_perf}
	% % \begin{tabular}{|c|c|c|c|c|c|}
	% % \hline
	% % \textbf{Dataset} & \textbf{Accuracy} & \textbf{FAR} & \textbf{FRR} & \textbf{GAR} & \textbf{GRR}  \\ \hline
	% %   SPD2010  & 94.2\% & 1.2\% & 10.45\%  & 89.6\% & 98.8\%\\ \hline
	% %   FVC2002 DB2  & 82.5\% & 5\% & 30\% & 70\%& 95\% \\ \hline
	% %   Synthetic  & 88.9\% & 1.6\% & 20.6\% & 79.4\%& 98.4\% \\ \hline
	% % \end{tabular}
	% % \end{table}

	% \begin{table}[H]
	% 	\centering
	% 	% \scriptsize
	% 	\caption{Fingerprint Recognition Performance}
	% 	\label{tab:fr_perf}

	% 	\setlength{\arrayrulewidth}{0.1mm}
	% 	\setlength{\tabcolsep}{5pt}
	% 	\renewcommand{\arraystretch}{1.1}

	% 	\begin{tabular}{|c|c|c|c|c|c|c|}
	% 		\hline
	% 		\textbf{Type} & \textbf{Dataset} & \textbf{Accuracy} & \textbf{FAR} & \textbf{FRR} & \textbf{GAR} & \textbf{GRR} \\ \hline
	% 		\multirow{2}{4em}{Without                                                                                        \\ROI} &
	% 		SPD2010       & 87.4\%           & 3.60\%            & 21.6\%       & 78.4\%       & 96.4\%                      \\ \cline{2-7}
	% 		              & FVC2002 DB2      & 58.7\%            & 15\%         & 67.5\%       & 32.5\%       & 85\%         \\ \cline{2-7}
	% 		\hline
	% 		\multirow{2}{5em}{With ROI}
	% 		              & SPD2010          & 94.2\%            & 1.2\%        & 10.45\%       & 89.6\%       & 98.8\%       \\ \cline{2-7}
	% 		              & FVC2002 DB2      & 82.5\%            & 5\%          & 30\%         & 70\%         & 95\%         \\ \cline{2-7}
	% 		\hline
	% 	\end{tabular}
	% \end{table}

	% \end{comment}

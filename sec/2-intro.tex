% % 1-2 page intro
\section{Introduction}
\label{sec:intro}
% % Basic Intro
\cch[Paragraph 1 (Context): Area of research, current trends/ issues/ limitations, challenges to overcome etc.]

% \begin{comment}
    % Fingerprint biometrics are extensively employed to identify individuals based on their distinct physical characteristics. 
    % This recognition method ensures permanence and distinctiveness, establishing a foundation for non-repudiation. 
    % While current recognition techniques are satisfactory, there is room for improvement by extracting a Region of Interest (ROI) that remains consistent across multiple impressions and exhibits unique distinguishing features.
    % By employing this ROI, computational resources are optimized, resulting in more efficient and faster fingerprint recognition systems.
    % Moreover, this extraction process helps mitigate challenges associated with noise, image quality, and partial acquisition.
    % The extracted ROI also contributes to reducing false acceptance or rejection rates and enhances compatibility and interoperability among different fingerprint recognition systems and databases.
    % Nevertheless, extracting the ROI from a fingerprint image remains a challenging task due to factors such as noise, quality, partial capture, etc.
    % 

    % Biometric-based authentication systems are becoming ubiquitous in our daily lives.
    % Among the various biometric modalities, fingerprint is one of the most popular and widely used modalities for personal identification and verification,
    % with applications ranging from access control to identity verification.
    % The uniqueness and permanence of fingerprints have made them a reliable means of identification.
    % %% talk about roi first
    % Extraction of the region of interest (ROI) is an essential component of fingerprint recognition systems, as it contains unique and distinctive fingerprint features that can be used for identification.
    % %% Why ROI
    % However, the process of extracting the ROI from a fingerprint image is a challenging task,
    % as it is often affected by several factors, such as noise, orientation, and variations in fingerprint patterns.
% \end{comment}

\cch[Paragraph 2 (Research Gap): Need of the day, Research gap to fulfil the need, motivation of the work]


\cch[Paragraph 3 (Scope and Objectives): Scope of work and objectives, What is the objective of this work?]
    
\cch[Paragraph 4 (Proposed Approach): Overview of the proposed methodology]
    
\cch[Paragraph 5 (Contribution of the work):]

\cch[Paragraph 6 (Organization of the paper):]
The rest of this paper is organized as follows.
Section~\ref{sec:related} provides an overview of the related work in the field of \cc[your area of research].
Section~\ref{sec:method} describes the proposed approach in detail.
Section~\ref{sec:results} presents the experimental results and analysis.
Finally, Section~\ref{sec:conclusion} concludes the paper and discusses future work.





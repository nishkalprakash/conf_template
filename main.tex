\documentclass[runningheads]{./lib/llncs}

% This is samplepaper.tex, a sample chapter demonstrating the
% LLNCS macro package for Springer Computer Science proceedings;
% Version 2.20 of 2017/10/04
% \usepackage{verbatim}

% uncomment to include stuff in standard comment-environment
% \includecomment{comment}
\usepackage{xcolor}
\usepackage{graphicx}
\usepackage{enumitem}
\usepackage{underscore} % to allow underscores in the doi
\usepackage{comment} % for multiline comments

% Used for displaying a sample figure. If possible, figure files should
% be included in EPS format.
\graphicspath{{./img/}{./tmp/}}

% If you use the hyperref package, please uncomment the following line
% to display URLs in blue roman font according to Springer's eBook style:
% \renewcommand\UrlFont{\color{blue}\rmfamily}


\newcommand\cc[1]{} % usage: \cc{In line comment}
\newcommand\ccut[1]{} % usage: \cccut{latex code that will be removed}
\newcommand\cch[1]{} % usage: \cch{Heading Comment}
\newcommand\cci[1]{} % usage: \cci{Comment Item}

\includecomment{draft}
% \excludecomment{draft}
\begin{draft}
	\renewcommand\cc[1]{\textcolor{blue}{#1}}
	\renewcommand\cch[1]{
		% \par\noindent
		\par\noindent
		% \break
		\cc{#1}
		\hfill\break
		% \par
	}
	\renewcommand\cci[1]{
		\par\noindent
		% \hfill\break
		\begin{itemize}[nosep]
			\nointerlineskip
			\item \cc{#1}
		\end{itemize}
	}

	\renewcommand\ccut[1]{
		\cc{\\\framebox[\textwidth]{$\downarrow$ START COMMENT $\downarrow$}}#1\cc{\\\framebox[\textwidth]{$\uparrow$ END COMMENT $\uparrow$}}
	}
\end{draft}

% % -------------- All packages go above this line --------------
\begin{document}

% \setlength{\textfloatsep}{1\baselineskip plus 0.2\baselineskip minus 0.5\baselineskip}
% \setlength{\parskip}{2pt plus2pt}
% \setlength{\abovedisplayskip}{0pt}
% \setlength{\belowdisplayskip}{0pt}
% \setlength{\floatsep}{1ex}
% \setlength{\abovecaptionskip}{1ex}
% \setlength{\belowcaptionskip}{1ex}
% \setlength{\intextsep}{1ex}

% ----------------- Title -----------------
% % Title
% \title{<Your contribution title>}
% All words in titles should be capitalized except for conjunctions, prepositions
% (e.g. on, of, by, and, or, but, from, with, without, under), and definite/indefinite
% articles (the, a, an), unless they appear at the beginning. Formula letters are
% typeset as in the text. Long titles that run over multiple lines can be wrapped
% explicitly with \\. Titles have no end punctuation.

% ------------------------- TITLE -------------------------
% title should be 10-12 words preferably
% \title{Indirect Feature Extraction for Efficient Fingerprint Recognition}
\title{Title Goes here}
%
%\titlerunning{Abbreviated paper title}
% If the paper title is too long for the running head, you can set
% an abbreviated paper title here
%
% ------------------------- AUTHORS -------------------------
\author{
    Nishkal Prakash
        \inst{1}
        \orcidID{0000-0002-1503-5754} 
\and
    Abhibhu Prakash
        \inst{1}
        \orcidID{0000-0000-0000-0000} 
\and
    Alka Ranjan
        \inst{1}
        \orcidID{0009-0005-4203-1643} 
\and
    Santhoshkumar Peddi
        \inst{1}
        \orcidID{0009-0001-7732-6637} 
\and
    Debasis Samanta
        \inst{1}
        \orcidID{0000-0002-6104-3771}
}
%
\authorrunning{Nishkal et al.}
% First names are abbreviated in the running head.
% If there are more than two authors, 'et al.' is used.
%
\institute{
    Indian Institute of Technology Kharagpur, \\
    Kharagpur-721302, West Bengal, India \\
    \email{
        \{nishkal,dsamanta\}@iitkgp.ac.in
    }
}


    % \title{Indirect Feature Extraction for Efficient Fingerprint Recognition}
    % %
    % %\titlerunning{Abbreviated paper title}
    % % If the paper title is too long for the running head, you can set
    % % an abbreviated paper title here
    % %
    % \author{Nishkal Prakash\inst{1}\orcidID{0000-0002-1503-5754} \and
    % Abhibhu Prakash\inst{1}\orcidID{0000-0000-0000-0000} \and
    % Alka Ranjan\inst{1}\orcidID{0009-0005-4203-1643} \and
    % Santhoshkumar Peddi\inst{1}\orcidID{0009-0001-7732-6637} \and
    % Debasis Samanta\inst{1}\orcidID{0000-0002-6104-3771}}
    % %
    % \authorrunning{Nishkal et al.}
    % % First names are abbreviated in the running head.
    % % If there are more than two authors, 'et al.' is used.
    % %
    % \institute{Indian Institute of Technology Kharagpur, \\
    % Kharagpur-721302, West Bengal, India \\
    % \email{\{nishkal,dsamanta\}@iitkgp.ac.in}}
    % \url{http://www.springer.com/gp/computer-science/lncs} \and
    % ABC Institute, Rupert-Karls-University Heidelberg, Heidelberg, Germany\\
    % \email{\{abc,lncs\}@uni-heidelberg.de}}
    %

\maketitle              % typeset the header of the contribution

% ----------------- Abstract -----------------
\begin{abstract}
	\cch{basic intro of topic 200-250 words}

	
	\cch{Context \{1-2 sentences\}}
	
	\cch{Major Research Gap \{1-2 sentences\}}

	\cch{Proposed technique/ Approach \{1-3 sentences\}}
	
	\cch{Contribution/Achievement of the work \{2-3 sentences\}}
	
	\keywords{
		First keyword
		\and
		Second keyword
		\and
		Another keyword
	}
\end{abstract}


% ----------------- Introduction -----------------
% % 1-2 page intro
\section{Introduction}
\label{sec:intro}
% % Basic Intro
\cch{Paragraph 1 (Context): Area of research, current trends/ issues/ limitations, challenges to overcome etc.}

% \begin{comment}
    % Fingerprint biometrics are extensively employed to identify individuals based on their distinct physical characteristics. 
    % This recognition method ensures permanence and distinctiveness, establishing a foundation for non-repudiation. 
    % While current recognition techniques are satisfactory, there is room for improvement by extracting a Region of Interest (ROI) that remains consistent across multiple impressions and exhibits unique distinguishing features.
    % By employing this ROI, computational resources are optimized, resulting in more efficient and faster fingerprint recognition systems.
    % Moreover, this extraction process helps mitigate challenges associated with noise, image quality, and partial acquisition.
    % The extracted ROI also contributes to reducing false acceptance or rejection rates and enhances compatibility and interoperability among different fingerprint recognition systems and databases.
    % Nevertheless, extracting the ROI from a fingerprint image remains a challenging task due to factors such as noise, quality, partial capture, etc.
    % 

    % Biometric-based authentication systems are becoming ubiquitous in our daily lives.
    % Among the various biometric modalities, fingerprint is one of the most popular and widely used modalities for personal identification and verification,
    % with applications ranging from access control to identity verification.
    % The uniqueness and permanence of fingerprints have made them a reliable means of identification.
    % %% talk about roi first
    % Extraction of the region of interest (ROI) is an essential component of fingerprint recognition systems, as it contains unique and distinctive fingerprint features that can be used for identification.
    % %% Why ROI
    % However, the process of extracting the ROI from a fingerprint image is a challenging task,
    % as it is often affected by several factors, such as noise, orientation, and variations in fingerprint patterns.
% \end{comment}

\cch{Paragraph 2 (Research Gap): Need of the day, Research gap to fulfil the need, motivation of the work}


\cch{Paragraph 3 (Scope and Objectives): Scope of work and objectives, What is the objective of this work?}
    
\cch{Paragraph 4 (Proposed Approach): Overview of the proposed methodology}
    
\cch{Paragraph 5 (Contribution of the work):}

\cch{Paragraph 6 (Organization of the paper):}
The rest of this paper is organized as follows.
Section~\ref{sec:related} provides an overview of the related work in the field of \cc{your area of research}.
Section~\ref{sec:method} describes the proposed approach in detail.
Section~\ref{sec:results} presents the experimental results and analysis.
Finally, Section~\ref{sec:conclusion} concludes the paper and discusses future work.






% ----------------- Related Work -----------------
% 1-2page Related work goes here

\section{Related Work}
\label{sec:related}

\cch[Explain the reason for doing this work, why is this problem important]

	% \begin{subsection}{ROI Extraction}
	% Fingerprint recognition has been widely studied, and various approaches have been proposed for ROI extraction.
	% Processing the entire fingerprint image is a computationally expensive task.
	% Extracting the ROI from the fingerprint image before processing it further is necessary.
	% A brief overview of some of the related works is presented in this section.
	%% using the entire fingerprint
	% Segmentation of fingerprint images into discreet blocks and then clustering based on a specific threshold value is a widely used approach \cite{wahhab2019novel,andrezza2021simple}.
	% One such work evaluated the area of interest based on a quality metric called "coherence" \cite{wahhab2019novel}, an experimental threshold value, 
	% to decide whether the block is in the foreground. 
	% The fixed block size and the specified threshold value are the limitations of the work.
	% %
	% Various approaches have been proposed in the past for extracting ROI (Region of Interest) from fingerprint images.
	% These approaches can be broadly classified into the following categories:
	% \begin{itemize}
	% 	\item \textbf{Segmentation-Based Approach:} This approach involves using techniques such as watershed segmentation \cite{sumijan2020fingerprint}, or active contour models \cite{hilles2021adaptive} to segment the fingerprint image and extract the ROI. However, this approach may suffer from over-segmentation or under-segmentation, which can affect the accuracy of ROI extraction.
	% 	\item \textbf{Filtering-Based Approach:} This approach involves using filters such as Sobel, Prewitt, or Canny \cite{sujatha2015performance}, to detect the edges in the fingerprint image, followed by thresholding \cite{lupu2014development} to extract the ROI. However, this approach may be affected by noise and variations in fingerprint patterns, which can limit its accuracy.
	% 	\item \textbf{Clustering-Based Approach:} This approach involves using clustering techniques such as K-means clustering \cite{mehdi2019improving} to group similar pixels in the fingerprint image and extract the ROI. However, the accuracy of clustering may be affected by the choice of initial centroids and the number of clusters.
	% 	\item \textbf{Hybrid Approaches:} These approaches combine the advantages of different methods, such as filtering and segmentation \cite{zhang2013adaptive}, to improve the accuracy and efficiency of ROI extraction. The proposed approach combines segmentation with filtering using statistical measures and agglomerative clustering to extract the ROI from fingerprint images.
	% 	\item \textbf{Deep Learning-Based Approaches:} In recent years, deep learning-based approaches \cite{joshi2021sensor,wan2020xfinger}, such as convolutional neural networks (CNNs) \cite{li2018fingerprint}, have been used for ROI extraction from fingerprint images. These approaches can achieve high accuracy but require large training data and computing resources.
	% \end{itemize}
	% In summary, there have been various approaches proposed in the past for ROI extraction from fingerprint images, each with its advantages and limitations. 
	% The proposed clustering-based hybrid approach aims to overcome the limitations of existing methods and provide a more accurate and efficient solution for ROI extraction.
	% The approach is computationally inexpensive and can be used for real-time applications while being able to handle noise and variations in fingerprint patterns.
	% This allows for generalizing the approach to a wide range of fingerprint images obtained from different sensors with different noise levels and variations in fingerprint patterns.
	%% Most basic approach
	% One popular approach for ROI extraction is based on filters. 
	% A filter is a mathematical operation that is applied to an image to enhance or extract certain features. In this case, the filter is used to extract the edges of the fingerprint image.
	% Sobel and Prewitt filters \cite{sujatha2015performance}, the most basic edge detection filters, have been used to detect edges in fingerprint images, 
	% and the resulting edge map is used to extract the ROI. 
	% However, the accuracy of edge detection can be affected by noise and variations in fingerprint patterns, which limits the performance of this approach.
	% To attenuate the effect of noise, work has been done to improve the performance of the Sobel filter by using a median filter \cite{zhang2013adaptive}.
	% The median filter is used to remove noise from the image, and the Sobel filter is used to extract the edges of the fingerprint image.
	% The resulting edge map is used to extract the ROI. The performance of this approach is better than that of the Sobel filter alone, but it is still prone to noise and variations in fingerprint patterns.
	% An approach for optimal filter obtained through convolution method, which was used for image enhancement and ROI extraction, was able to extract the ROI with high accuracy  with the only drawback being that the approach was computationally expensive.
	% \end{subsection}

\cch[How was it done in the very beginning]
	\cc[Try to cite \cite{peddi2023ROIClustering} here]
\cch[How much work has been done since then]
	% Standard deviation, median and binarization filter to identify area of interest using parameters like median window size, binarization threshold and the sizes
	% of the inner block and outer block also have been used \cite{andrezza2021simple}. These parameters had experimental threshold values, which were not optimal for all the images.

\cch[What are the research Gaps]

\cch[How this work is different from the existing work]
	% The proposed approach uses a combination of statistical measures with simple clustering to extract the ROI from the fingerprint image.

\cch[Background for the proposed work]
% A brief description of the background/preliminaries of the work. 
% Briefly to the point and precisely (1-2 pages).









% ----------------- Methodology -----------------
\section{Proposed method}
\label{sec:method}
\cch[4-6 page proposed method]


\cch[What problems(s) have been addressed in this work?]

\cch[Overview of the proposed approach (try to put graphical abstracts)]

\cch[Sub-section-wise detail description of the substeps involved.]

\cch[Take care of the following:]
\cci[What is the problem and what is its input]

\cci[How have you solved the problem?]

\cci[Why have you proposed this solution?]


	% \begin{comment}
	% \setlength{\belowdisplayskip}{5pt} \setlength{\belowdisplayshortskip}{2pt}
	% \setlength{\abovedisplayskip}{2pt} \setlength{\abovedisplayshortskip}{0pt}
	% The proposed method (shown in figure \ref{fig:overview}) involves dividing the fingerprint image into blocks of equal size. These blocks are used as the basis for feature extraction, which is the process of identifying and isolating essential characteristics of the image. Once the features have been extracted, they are grouped into three clusters. These clusters represent different parts of the fingerprint image that share similar characteristics, and one of them is mapped to the fingerprint image to form the ROI.
	% \begin{figure}[H]
	% 	\centering
	% 	\includegraphics[width=\textwidth]{images/overview.png}
	% 	\caption{The overview of the proposed method}
	% 	\label{fig:overview}
	% \end{figure}

	% The method involves five key steps: block segmentation of the fingerprint image, feature extraction from each block,  feature clustering based on the distance matrix, identification of a target cluster and mapping the target cluster to a region in the fingerprint image, which is considered as the ROI.

	% \subsection{Block Segmentation}

	% The block segmentation involves dividing an input fingerprint image $I$ of size $m \times n$ into non-overlapping blocks of size $p \times p$, where $p$ is the size of the blocks.
	% \begin{equation} \label{eqn:block}
	% 	B_{i,j} = I_{(i-1)\times p+1:i\times p, (j-1)\times p+1:j\times p}
	% \end{equation}
	% As shown in the equation \ref{eqn:block}, the $i, j$th block is denoted by $B_{i,j}$, and is obtained by extracting the pixels in rows $(i-1)\times p+1$ to $i \times p$ and columns $(j-1)\times p+1$ to $j \times p$ from the original image $I$. The resulting block matrix $B$ is of size $(m/p) \times (n/p)$.

	% \subsection{Feature Extraction} \label{feature}

	% For each block $B_{i,j}$ of size $p \times p$, the five features $F_{i,j}^1$, $F_{i,j}^2$, $F_{i,j}^3$, $F_{i,j}^4$, and $F_{i,j}^5$ are calculated, which represent the gradient, ridge orientation, curvature, Hessian and ridge frequency features, respectively.

	% Then, the mean of these five features is taken to form a 5-dimensional feature vector for the block $B_{i,j}$, which is denoted as $V(B_{i,j})$ and is defined as:
	% $$V(B_{i,j}) = [\text{mean}(F_{i,j}^1), \text{mean}(F_{i,j}^2), \text{mean}(F_{i,j}^3), \text{mean}(F_{i,j}^4), \text{mean}(F_{i,j}^5)]$$
	% % \begin{equation}
	% % V = [V(B_{i,j})]_{i=1,\ldots,m/p,,j=1,\ldots,n/p}
	% % \end{equation}
	% % where each V(Bi,j) is a 5-dimensional feature vector representing the i, jth block.

	% In other words, the 5-dimensional feature vector $V(B_{i,j})$ is formed by taking the mean of the five features calculated for the block $B_{i,j}$. The feature vector is generated for each block in the fingerprint image to obtain a set of feature vectors representing the entire fingerprint.

	% The gradient features $F_{i,j}^1$ of the blocks are obtained using the Sobel operator, as shown below:
	% \begin{equation}
	% 	\openup\jot
	% 	\begin{aligned}[t]
	% 		F_{i,j}^1 & = \sqrt{G_x^2 + G_y^2}
	% 	\end{aligned}
	% 	\begin{aligned}[t]
	% 		\text{, where\quad}
	% 		G_x & = B_{i,j} * K_x
	% 		\text{,}
	% 	\end{aligned}
	% 	\begin{aligned}[t]
	% 		G_y & = B_{i,j} * K_y
	% 	\end{aligned}
	% \end{equation}




	% % \begin{equation} \label{grad}
	% %     \begin{matrix}
	% %         G_x = B_{i,j} * K_x, G_y = B_{i,j} * K_y, F_{i,j}^1 = \sqrt{G_x^2 + G_y^2}  \\[0.5cm]
	% %     \end{matrix}
	% % \end{equation}

	% Here, $K_x$ and $K_y$ are Sobel operator kernels for the X and Y direction.

	% The ridge orientation feature $F_{i,j}^2$ captures the direction of the fingerprint ridges at each pixel in the block $B_{i,j}$. A gradient-based approach is used to extract the direction of the ridges in the blocks using the equations,

	% \begin{equation} \label{orient}
	% 	\begin{matrix}
	% 		F_{i,j}^2 = \displaystyle\frac{\pi}{2} + \displaystyle\frac{\arctan(2G_{xy}/(G_{xx} - G_{yy}))}{2} \\[0.1cm]
	% 		\text{where,\quad $G_{xx} = \sum_{B} {G_x^2}$,  $G_{yy} = \sum_B {G_y^2}$ and $G_{xy} = \sum_B {G_xG_y}$}
	% 	\end{matrix}
	% \end{equation}

	% The local curvature $F_{i,j}^3$ of the ridges and the ridge frequency $F_{i,j}^5$ are calculated using the Hessian matrix $F_{i,j}^4$ and second-order gradients, as shown in the equation below. The Hessian matrix is also considered a separate feature.
	% \begin{equation}
	% 	\openup\jot % make lines a little more far apart
	% 	\begin{aligned}[t]
	% 		F_{i,j}^3 & = \displaystyle\frac{F_{i,j}^4}{(1+G_x^2+G_y^2)^2}
	% 	\end{aligned}
	% 	\qquad % adjust to suit
	% 	\begin{aligned}[t]
	% 		F_{i,j}^5 & = \displaystyle\frac{1}{2\pi}\sqrt{\frac{\lambda_1 \lambda_2}{\lambda_1 + \lambda_2}}
	% 	\end{aligned}
	% \end{equation}
	% where, $\lambda_1$ and $\lambda_2$ are the eigenvalues of the Hessian matrix and $F_{i,j}^4 = (G_{xx} \cdot G_{yy}) - (G_{xy} \cdot G_{yx})$.
	% % \begin{equation} \label{curv}
	% %     \begin{matrix}
	% %         F_{i,j}^3 = \displaystyle\frac{F_{i,j}^4}{(1+G_x^2+G_y^2)^2}\\[0.5cm] F_{i,j}^5 = \displaystyle\frac{1}{2\pi}\sqrt{\frac{\lambda_1 \lambda_2}{\lambda_1 + \lambda_2}}\\[0.5cm]
	% %         \text{where, $\lambda_1$ and $\lambda_2$ are the eigenvalues of the Hessian matrix and } \\ F_{i,j}^4 = (G_{xx} \cdot G_{yy}) - (G_{xy} \cdot G_{yx})

	% %     \end{matrix}    
	% % \end{equation}
	% \subsection{Agglomerative clustering}
	% After extracting the feature vectors for each block, a distance matrix is calculated based on the Euclidean distance. The Euclidean distance between two feature vectors $V_i$ and $V_j$  is defined as $d_{ij}$ and the distance matrix as $D$.
	% % \begin{equation}
	% % 	\openup\jot % 
	% % 	\begin{aligned}[t]
	% % 		D = \begin{bmatrix}
	% % 			    d_{1,1} & d_{1,2} & \cdots & d_{1,n} \\
	% % 			    d_{2,1} & d_{2,2} & \cdots & d_{2,n} \\
	% % 			    \vdots  & \vdots  & \ddots & \vdots  \\
	% % 			    d_{n,1} & d_{n,2} & \cdots & d_{n,n}
	% % 		    \end{bmatrix}
	% % 	\end{aligned}
	% % 	\qquad % adjust to suit
	% % 	\begin{aligned}[t]
	% % 		d_{ij} & = \sqrt{\sum_{k=1}^{5} (V_{i,k} - V_{j,k})^2}
	% % 	\end{aligned}
	% % \end{equation}

	% % \begin{equation}
	% % d_{ij} = \sqrt{\sum_{k=1}^{5} (V_{i,k} - V_{j,k})^2}
	% % \end{equation}
	% % In mathematical notation, the distance matrix $D$ can be represented as:
	% % \begin{equation}
	% % D = \begin{bmatrix}
	% % d_{1,1} & d_{1,2} & \cdots & d_{1,n} \\
	% % d_{2,1} & d_{2,2} & \cdots & d_{2,n} \\
	% % \vdots & \vdots & \ddots & \vdots \\
	% % d_{n,1} & d_{n,2} & \cdots & d_{n,n}
	% % \end{bmatrix}
	% % \end{equation}

	% By using this distance matrix D, agglomerative clustering is used to form the clusters. Agglomerative clustering is a hierarchical clustering technique that starts with each vector $V$ as a separate cluster and iteratively merges the most similar cluster until the desired number of clusters is obtained, which, in this case, is three. These three clusters are assumed to be the background, ROI, and other parts of the fingerprint image. Here, the distance between clusters is defined as the minimum distance between any two feature vectors belonging to different clusters. Then, the two closest clusters are merged, and the distance matrix is updated accordingly. This process is repeated until the desired number of clusters is obtained.

	% \subsection{Identification of Target cluster}
	% % Once the agglomerative clustering algorithm generates three clusters, they can be further refined by selecting a specific cluster to represent the ROI in the fingerprint image. To do this, the cosine similarity $CS$ can be calculated between the feature vectors within each cluster and the cluster with the highest average cosine similarity $ACS(C_i)$ can be selected. 
	% Once the agglomerative clustering algorithm generates three clusters, a specific cluster is selected to represent the ROI in the fingerprint image. This selection is based on calculating the cosine similarity (CS) between the feature vectors within each cluster, with the cluster displaying the highest average cosine similarity (ACS(Ci)) being chosen. To account for background similarity, additional metrics like curvature and orientation are employed. Among the remaining two clusters, the one with more dissimilarities in the edge regions, where ridges may terminate or exhibit varying orientations, is disregarded. Instead, priority is given to the cluster that shows greater consistency and similarity in terms of ridge orientation and curvature. This meticulous approach ensures accurate identification of the desired ROI.The cosine similarity between two feature vectors $V_i$ and $V_j$ can be computed as follows:
	% \begin{equation}
	% 	\openup\jot % make lines a little more far apart
	% 	\begin{aligned}[t]
	% 		ACS(C_i) &= \frac{\sum_{i,j} CS(V_i, V_j)}{N(N-1)/2}
	% 	\end{aligned}
	% 	\qquad % adjust to suit
	% 	\begin{aligned}[t]
	% 			CS(V_i, V_j) &= \frac{V_i \cdot V_j}{\lVert V_i \rVert \cdot \lVert V_j \rVert}
	% 	\end{aligned}
	% \end{equation}
	% % \begin{equation}
	% % 	CS(V_i, V_j) = \frac{V_i \cdot V_j}{\lVert V_i \rVert \cdot \lVert V_j \rVert} = \frac{\sum_{k=1}^{5} V_{i,k} \cdot V_{j,k}}{\sqrt{\sum_{k=1}^{5} V_{i,k}^2} \cdot \sqrt{\sum_{k=1}^{5} V_{j,k}^2}}
	% % \end{equation}
	% % \begin{equation}
	% % 	ACS(C_i) = \frac{\sum_{i,j} CS(V_i, V_j)}{N(N-1)/2}
	% % \end{equation}
	% Here, $V_i \cdot V_j$ denotes the dot product of the feature vectors, and $\lVert V_i \rVert$, $\lVert V_j \rVert$ represent their Euclidean norms, $C_i$ represents the $i$-th cluster, $N$ represents the number of feature vectors in the cluster, and the sum is taken over all unique pairs of feature vectors within the cluster. Finally, the cluster with the highest average cosine similarity is selected as the Target cluster.

	% \subsection{Identification of ROI}
	% After selecting the target cluster with the highest average cosine similarity, it can be mapped back to the original fingerprint image to identify the ROI by finding the corresponding block indices of features in the cluster. To calculate the block indices i,j the following formulae have been used.
	% \begin{equation*}
	% 	\openup\jot % make lines a little more far apart
	% 	\begin{aligned}[t]
	% 		i & = \left\lfloor \frac{(k-1)}{p} \right\rfloor + 1
	% 	\end{aligned}
	% 	\qquad % adjust to suit
	% 	\begin{aligned}[t]
	% 		j & = (k-1) \bmod p + 1
	% 	\end{aligned}
	% \end{equation*}
	% % \begin{equation}
	% % i = \left\lfloor \frac{(k-1)}{p} \right\rfloor + 1
	% % \end{equation}

	% % \begin{equation}
	% % j = (k-1) \bmod p + 1
	% % \end{equation}
	% where $k$ is the index of the feature vector in an array of feature vectors.

	% Then each block is mapped to the original image by calculating the top left ($r,c$) and bottom right ($r^{'},c^{'}$) indices of that block in the original image.
	% \begin{equation*}
	% 	\openup\jot
	% 	\begin{aligned}[t]
	% 		r & = (i-1) \times p + 1
	% 		\text{,}
	% 	\end{aligned}
	% 	\quad
	% 	\begin{aligned}[t]
	% 		c & = (j-1) \times p + 1
	% 		\text{,}
	% 	\end{aligned}
	% 	\quad
	% 	\begin{aligned}[t]
	% 		r^{'} & = i \times p
	% 		\text{,}
	% 	\end{aligned}
	% 	\quad
	% 	\begin{aligned}[t]
	% 		c^{'} & = j \times p
	% 	\end{aligned}
	% \end{equation*}

	% % \begin{equation*}
	% % \openup\jot 
	% % \begin{aligned}[t]
	% % r' &= i \times p
	% % \end{aligned}
	% % \qquad % adjust to suit
	% % \begin{aligned}[t]
	% %  c' &= j \times p
	% % \end{aligned}
	% % \end{equation*}

	% % Starting row index:
	% % \begin{equation}
	% % r = (i-1) \times p + 1
	% % \end{equation}

	% % Starting column index:
	% % \begin{equation}
	% % c = (j-1) \times p + 1
	% % \end{equation}

	% % Ending row index:
	% % \begin{equation}
	% % r' = i \times p
	% % \end{equation}

	% % Ending column index:
	% % \begin{equation}
	% % c' = j \times p
	% % \end{equation}

	% After mapping the block indices to the original fingerprint, the next step is to determine the coordinates of the bounding box that encloses the target cluster. This is done by comparing the indices of all the blocks that are selected for the cluster and finding the minimum and maximum values for the row and column indices. The top left corner of the bounding box is given by the minimum row and column indices, while the bottom right corner is given by the maximum row and column indices. Once these four indices are determined, a bounding box is drawn using them to indicate the ROI.
	% \end{comment}




% ----------------- Experiments -----------------
% 1 page results
\section{Experimental Results}
\label{sec:results}

\{Detail design of experiment/setup/protocols/subjects/data collection/ or a detailed description of the used datasets\}

\{Objectives of the experiment/what do you want to prove/substantiate?\}

\{Experimental Environment, with a link to code/data repository\}

\{Result vis-a-vis experimental objectives with proper tables, graphs, plots, etc\}


	% \begin{comment}
	
	% \setlength{\belowdisplayskip}{5pt} \setlength{\belowdisplayshortskip}{2pt}
	% \setlength{\abovedisplayskip}{2pt} \setlength{\abovedisplayshortskip}{0pt}
	% For this experiment, two datasets, FVC2002 DB2 and SPD2010, have been used to check the region consistency and prove the core point's existence in the ROI, respectively.For the block size, statistical metrics, and clustering methods both datasets are used combinedly.

	% % \subsubsection{Region Consistency:}
	% \textbf{\textit{Region Consistency:}}
	% To prove that the region obtained is consistent across all impressions of the same fingerprint, a fingerprint is considered from the FVC dataset. The ROI computed for these fingerprints is shown in figure \ref{fig:roi_fvc}, and it can be seen that the ROI is consistent.

	% \begin{figure}[H]
	% 	\begin{subfigure}[b]{0.12\textwidth}
	% 		\centering
	% 		\includegraphics[width=\textwidth]{images/fvc12.png}
	% 		% \caption{ }
	% 		\label{conf_1}
	% 	\end{subfigure}
	% 	\hfill
	% 	\begin{subfigure}[b]{0.12\textwidth}
	% 		\centering
	% 		\includegraphics[width=\textwidth]{images/fvc13.png}
	% 		% \caption{ }
	% 		\label{conf_2}
	% 	\end{subfigure}
	% 	\hfill
	% 	\begin{subfigure}[b]{0.12\textwidth}
	% 		\centering
	% 		\includegraphics[width=\textwidth]{images/fvc14.png}
	% 		% \caption{ }
	% 		\label{conf_3}
	% 	\end{subfigure}
	% 	\hfill
	% 	\begin{subfigure}[b]{0.12\textwidth}
	% 		\centering
	% 		\includegraphics[width=\textwidth]{images/fvc15.png}
	% 		% \caption{ }
	% 		\label{conf_3}
	% 	\end{subfigure}
	% 	\hfill
	% 	\begin{subfigure}[b]{0.12\textwidth}
	% 		\centering
	% 		\includegraphics[width=\textwidth]{images/fvc16.png}
	% 		% \caption{ }
	% 		\label{conf_3}
	% 	\end{subfigure}
	% 	\hfill
	% 	\begin{subfigure}[b]{0.12\textwidth}
	% 		\centering
	% 		\includegraphics[width=\textwidth]{images/fvc_11.png}
	% 		% \caption{ }
	% 		\label{conf_3}
	% 	\end{subfigure}
	% 	% \begin{subfigure}[b]{0.12\textwidth}
	% 	%      \centering
	% 	%      \includegraphics[width=\textwidth]{images/syn_11.png}
	% 	%      \caption{ }
	% 	%      \label{conf_1}
	% 	%  \end{subfigure}
	% 	%  \hfill
	% 	%  \begin{subfigure}[b]{0.12\textwidth}
	% 	%      \centering
	% 	%      \includegraphics[width=\textwidth]{images/syn_12.png}
	% 	%      \caption{ }
	% 	%      \label{conf_2}
	% 	%  \end{subfigure}
	% 	%  \hfill
	% 	% \begin{subfigure}[b]{0.12\textwidth}
	% 	%      \centering
	% 	%      \includegraphics[width=\textwidth]{images/syn_13.png}
	% 	%      \caption{ }
	% 	%      \label{conf_3}
	% 	%  \end{subfigure}
	% 	%  \hfill
	% 	%  \begin{subfigure}[b]{0.12\textwidth}
	% 	%      \centering
	% 	%      \includegraphics[width=\textwidth]{images/syn_14.png}
	% 	%      \caption{ }
	% 	%      \label{conf_3}
	% 	%  \end{subfigure}
	% 	%  \hfill
	% 	%  \begin{subfigure}[b]{0.12\textwidth}
	% 	%      \centering
	% 	%      \includegraphics[width=\textwidth]{images/syn_15.png}
	% 	%      \caption{ }
	% 	%      \label{conf_3}
	% 	%  \end{subfigure}
	% 	% \hfill
	% 	%  \begin{subfigure}[b]{0.12\textwidth}
	% 	%      \centering
	% 	%      \includegraphics[width=\textwidth]{images/syn_16.png}
	% 	%      \caption{ }
	% 	%      \label{conf_3}
	% 	%  \end{subfigure}

	% 	\caption{ROI obtained for different impressions of a fingerprint.}
	% 	\label{fig:roi_fvc}
	% \end{figure}

	% \textit{\textbf{ROI with Corepoint:}}
	% As the SPD dataset contains the core point labels, it has been used to show that the ROI always contains a core point. Six fingerprints are selected at random, and their core points and ROI are shown in figure \ref{fig:core_spd}.

	% \begin{figure}[H]
	% 	\begin{subfigure}[b]{0.12\textwidth}
	% 		\centering
	% 		\includegraphics[width=\textwidth]{images/spd1.png}
	% 		% \caption{ }
	% 		\label{conf_1}
	% 	\end{subfigure}
	% 	\hfill
	% 	\begin{subfigure}[b]{0.12\textwidth}
	% 		\centering
	% 		\includegraphics[width=\textwidth]{images/spd2.png}
	% 		% \caption{ }
	% 		\label{conf_2}
	% 	\end{subfigure}
	% 	\hfill
	% 	\begin{subfigure}[b]{0.12\textwidth}
	% 		\centering
	% 		\includegraphics[width=\textwidth]{images/spd3.png}
	% 		% \caption{ }
	% 		\label{conf_3}
	% 	\end{subfigure}
	% 	\hfill
	% 	\begin{subfigure}[b]{0.12\textwidth}
	% 		\centering
	% 		\includegraphics[width=\textwidth]{images/spd4.png}
	% 		% \caption{ }
	% 		\label{conf_3}
	% 	\end{subfigure}
	% 	\hfill
	% 	\begin{subfigure}[b]{0.12\textwidth}
	% 		\centering
	% 		\includegraphics[width=\textwidth]{images/spd5.png}
	% 		% \caption{ }
	% 		\label{conf_3}
	% 	\end{subfigure}
	% 	\hfill
	% 	\begin{subfigure}[b]{0.12\textwidth}
	% 		\centering
	% 		\includegraphics[width=\textwidth]{images/spd6.png}
	% 		% \caption{ }
	% 		\label{conf_3}
	% 	\end{subfigure}

	% 	%  \begin{subfigure}[b]{0.12\textwidth}
	% 	%      \centering
	% 	%      \includegraphics[width=\textwidth]{images/spd7.png}
	% 	%      \caption{ }
	% 	%      \label{conf_1}
	% 	%  \end{subfigure}
	% 	%  \hfill
	% 	%  \begin{subfigure}[b]{0.12\textwidth}
	% 	%      \centering
	% 	%      \includegraphics[width=\textwidth]{images/spd8.png}
	% 	%      \caption{ }
	% 	%      \label{conf_2}
	% 	%  \end{subfigure}
	% 	%  \hfill
	% 	% \begin{subfigure}[b]{0.12\textwidth}
	% 	%      \centering
	% 	%      \includegraphics[width=\textwidth]{images/spd9.png}
	% 	%      \caption{ }
	% 	%      \label{conf_3}
	% 	%  \end{subfigure}
	% 	%  \hfill
	% 	%  \begin{subfigure}[b]{0.12\textwidth}
	% 	%      \centering
	% 	%      \includegraphics[width=\textwidth]{images/spd10.png}
	% 	%      \caption{ }
	% 	%      \label{conf_3}
	% 	%  \end{subfigure}
	% 	%  \hfill
	% 	%  \begin{subfigure}[b]{0.12\textwidth}
	% 	%      \centering
	% 	%      \includegraphics[width=\textwidth]{images/spd11.png}
	% 	%      \caption{ }
	% 	%      \label{conf_3}
	% 	%  \end{subfigure}
	% 	% \hfill
	% 	%  \begin{subfigure}[b]{0.12\textwidth}
	% 	%      \centering
	% 	%      \includegraphics[width=\textwidth]{images/spd12.png}
	% 	%      \caption{ }
	% 	%      \label{conf_3}
	% 	%  \end{subfigure}
	% 	\caption{A set of images from the SPD dataset with their ROI and core point.}
	% 	\label{fig:core_spd}
	% \end{figure}
	% \textit{\textbf{Block Size:}}
	% Based on the observation of the ROI generated by the method, the corresponding block size is given a score of `1' if the ROI contains the core point, and a `0' otherwise. Thus, the scores for different block sizes are compared, and the block sizes within the range of 16 to 35 have a score of `1'. A block size of 23 is chosen for the method as it is close to the mid-point of the range of the optimal block sizes.

	% % \begin{figure}[h!]
	% %     \centering
	% %     \includegraphics[width=0.5\textwidth]{images/block_size.jpg}
	% %     \caption{Variation of the score with respect to block size.}
	% %     \label{fig:block_size}
	% % \end{figure}

	% %A block size of 23 is chosen for the method as it is close to the mid-point of the range of the optimal block sizes.

	% \textit{\textbf{Statistical Metrics:}} Different statistical metrics like mean, variance, median and mode are used to reshape features in section \ref{feature}. The features obtained from these statistical metrics are checked for their miss rate, and it can be seen from figure \ref{fig:miss_rate} that the mean of features has the least amount of miss. For this study, it is considered to be a miss whenever the ROI doesn't contain the core point.

	% \textit{\textbf{Clustering Methods:}} The features are clustered based on the k-means and agglomerative algorithms. The Silhouette scores for the k-means and agglomerative clustering algorithms are computed, and as can be seen in figure \ref{fig:clus_score}, agglomerative clustering has the best score for the three clusters.
	% \begin{figure}[H]
	% 	\begin{subfigure}[b]{0.45\textwidth}
	% 		\centering
	% 		\includegraphics[width=\textwidth]{images/chart(5).png}
	% 		% \caption{Miss rate calculated for SPD2010}
	% 		\caption{Miss Rate}
	% 		\label{fig:miss_rate}
	% 	\end{subfigure}
	% 	\hfill
	% 	\begin{subfigure}[b]{0.45\textwidth}
	% 		\centering
	% 		\includegraphics[width=\textwidth]{images/chart(6).png}
	% 		% \caption{Silhouette score for the clustering algorithms}
	% 		\caption{Silhouette Score}
	% 		\label{fig:clus_score}
	% 	\end{subfigure}
	% 	\hfill
	% 	\caption{Analysis of the Miss Rate and Silhouette Score.}
	% 	\label{fig:fr_perf}
	% \end{figure}
	% % \vspace{-1.5em}
	% \textit{\textbf{Fingerprint Recognition:}} The evaluation of fingerprint recognition accuracy involves using half of the data from each dataset as genuine data and the other half as imposter data. A random fingerprint is selected from either half of the data and compared with all fingerprints from the genuine half. Based on the Harris corner matching score \cite{derpanis2004harris}, the selected fingerprint is declared to either have a match or no match. The performance of the method for a set of fingerprints with and without ROI is shown in figure \ref{fig:fr_perf}.

	% \begin{figure}[H]
	% 	\begin{subfigure}[b]{0.5\textwidth}
	% 		\centering
	%    \includegraphics[width=\textwidth]{images/chart(7).png}
	% 		% \includegraphics[width=\textwidth]{images/Updated_1.png}
	% 		\caption{Without ROI}
	% 		\label{conf_1}
	% 	\end{subfigure}
	% 	\hfill
	% 	\begin{subfigure}[b]{0.45\textwidth}
	% 		\centering
	% 		% \includegraphics[width=\textwidth]{images/updated_2.png}
	%   \includegraphics[width=\textwidth]{images/chart(8).png}
	% 		\caption{With ROI}
	% 		\label{conf_2}
	% 	\end{subfigure}
	% 	\hfill
	% 	\caption{Performance of the fingerprint recognition method}
	% 	\label{fig:fr_perf}
	% \end{figure}
	% Additionally, the performance of the fingerprint recognition method in terms of FAR, FRR, GAR and GRR is tabulated in table \ref{tab:fr_perf}. It can be seen that the performance of the method is better when the fingerprint is used along with an ROI.

	% \textit{\textbf{Comaprison of Time and Accuracy:}}
	% The average identification time is compared between the full fingerprint and the ROI in the dataset, as shown in Figure \ref{time_Comparison}. Additionally, Figure \ref{Acc_Comp} provides a comparison of the proposed method's accuracy with existing Cluster-based \cite{mehdi2019improving} and Filter-based \cite{841531} methods.

	% \begin{figure}[H]
	% 	\begin{subfigure}[b]{0.45\textwidth}
	% 		\centering
	%    \includegraphics[width=\textwidth]{images/premi_time_Comparison.png}
	% 		% \includegraphics[width=\textwidth]{images/Updated_1.png}
	% 		\caption{Time Comparison}
	% 		\label{time_Comparison}
	% 	\end{subfigure}
	% 	\hfill
	% 	\begin{subfigure}[b]{0.45\textwidth}
	% 		\centering
	% 		% \includegraphics[width=\textwidth]{images/updated_2.png}
	%   \includegraphics[width=\textwidth]{images/Comparison(1).png}
	% 		\caption{Accuracy comparison}
	% 		\label{Acc_Comp}
	% 	\end{subfigure}
	% 	\hfill
	% 	\caption{Time comparison and Accuracy comparison}
	% 	\label{fig:fr_perf}
	% \end{figure}

	% % \begin{table}[h]
	% % \setlength{\arrayrulewidth}{0.1mm}
	% % \setlength{\tabcolsep}{5pt}
	% % \renewcommand{\arraystretch}{1.5}
	% % \centering
	% % \caption{Fingerprint Recognition Performance with ROI}
	% % \label{tab:fr_perf}
	% % \begin{tabular}{|c|c|c|c|c|c|}
	% % \hline
	% % \textbf{Dataset} & \textbf{Accuracy} & \textbf{FAR} & \textbf{FRR} & \textbf{GAR} & \textbf{GRR}  \\ \hline
	% %   SPD2010  & 94.2\% & 1.2\% & 10.45\%  & 89.6\% & 98.8\%\\ \hline
	% %   FVC2002 DB2  & 82.5\% & 5\% & 30\% & 70\%& 95\% \\ \hline
	% %   Synthetic  & 88.9\% & 1.6\% & 20.6\% & 79.4\%& 98.4\% \\ \hline
	% % \end{tabular}
	% % \end{table}

	% \begin{table}[H]
	% 	\centering
	% 	% \scriptsize
	% 	\caption{Fingerprint Recognition Performance}
	% 	\label{tab:fr_perf}

	% 	\setlength{\arrayrulewidth}{0.1mm}
	% 	\setlength{\tabcolsep}{5pt}
	% 	\renewcommand{\arraystretch}{1.1}

	% 	\begin{tabular}{|c|c|c|c|c|c|c|}
	% 		\hline
	% 		\textbf{Type} & \textbf{Dataset} & \textbf{Accuracy} & \textbf{FAR} & \textbf{FRR} & \textbf{GAR} & \textbf{GRR} \\ \hline
	% 		\multirow{2}{4em}{Without                                                                                        \\ROI} &
	% 		SPD2010       & 87.4\%           & 3.60\%            & 21.6\%       & 78.4\%       & 96.4\%                      \\ \cline{2-7}
	% 		              & FVC2002 DB2      & 58.7\%            & 15\%         & 67.5\%       & 32.5\%       & 85\%         \\ \cline{2-7}
	% 		\hline
	% 		\multirow{2}{5em}{With ROI}
	% 		              & SPD2010          & 94.2\%            & 1.2\%        & 10.45\%       & 89.6\%       & 98.8\%       \\ \cline{2-7}
	% 		              & FVC2002 DB2      & 82.5\%            & 5\%          & 30\%         & 70\%         & 95\%         \\ \cline{2-7}
	% 		\hline
	% 	\end{tabular}
	% \end{table}

	% \end{comment}


% ----------------- Conclusion -----------------
% 1 page conclusion and future work and references
\section{Conclusion and Future Work}
\label{sec:conclusion}

\{Novelty of the work\}
    % The proposed method finds an ROI from a fingerprint image using agglomerative clustering, whereas existing methods crop a region from fingerprint images based on the threshold value or use morphological operations.

\{Metionable achievement/breakthrough\}

    % The proposed method is always able to consistently extract the ROI from the fingerprint in a relatively short amount of time.

\{Research application/potential\}

    % This method gives consistent ROI for fingerprints obtained from different scanners/sensors while also being translation and rotation invariant.
    
\{Claims if under some assumption\}
    
\{Limitations of the work\}
    % The proposed method performs better if a core point exists in the fingerprint.
    % The experiment has achieved good results for the three datasets and has successfully extracted a consistent ROI from each fingerprint.
    
\{Future scope or Future work\}
    % The work uses the Sobel filter to extract initial features from the entire fingerprint image. This can be improved by using a different filter or a machine-learning approach. The filter used in this experiment was able to give good matching accuracy. The use of a different clustering approach may also be explored.

    % The proposed method extracts the ROI from a fingerprint image using agglomerative clustering, surpassing existing methods that use threshold values or morphological operations for region cropping. It consistently and efficiently extracts the ROI, regardless of scanner/sensor variations, while maintaining translation and rotation invariance. The experiment successfully extracted the ROI for each fingerprint in the two datasets and achieved notable improvements in fingerprint recognition accuracy: 6.8\% for SPD2010 and 24.4\% for FVC2002. Moreover, the proposed method significantly reduced processing time by 80.75\% for SPD2010 and 59.29\% for FVC2002. To further enhance the method, alternative filters or machine learning approaches can be explored, and different clustering approaches could be investigated.

% \section{Research Highlights}
% Give 4-5 salient points of your research work, each in 10 words.

% ----------------- Bibliography -----------------
% % BibTeX users should specify bibliography style 'splncs04'.
% % References will then be sorted and formatted in the correct style.

\cci{Follow a bibliography style}
\cci{All recent/relevant papers only}
\cci{20-40 references}

\bibliographystyle{lib/splncs04}
\bibliography{sec/ref}

% \bibliographystyle{splncs04}
% \bibliography{mybibliography}


\end{document}

\documentclass{article}
% \documentclass[12pt,fleqn]{report} 
\usepackage[margin=1.5cm]{geometry}
\usepackage{graphicx} % Required for inserting images

%% Language and font encodings
\usepackage[english]{babel}
\usepackage[utf8x]{inputenc}

\Huge{\title{Format of Research Paper Writing}}
\author{Prof. Debasis Samanta}
\date{\vspace{-1ex}}

\begin{document}

\maketitle

\section{Outline of Paper}
\Large{\textbf{\underline{Title}}} :  \ldots 10-12 words preferably \ldots \\\\
\Large{\textbf{\underline{Abstract}}} : 200-250 words
\begin{itemize}
    \item Context (1-2 sentences)
    \item Major Research Gap (1-2 sentences)
    \item Proposed Technique/Approach (1-3 sentences)
    \item Contribution/Achievement of the work (2-3 sentences)
\end{itemize}
\Large{\textbf{\underline{Keywords}}} : 5-6 Keywords\\\\
\Large{\textbf{\underline{Introduction}}}\\\\
\Large{\textbf{\underline{Literature Survey}}}\\\\
\Large{\textbf{\underline{Background/Preliminaries}}}\\\\
\Large{\textbf{\underline{Proposed Methodology}}}\\\\
\Large{\textbf{\underline{Experiment / Experimental Results}}}\\\\
\Large{\textbf{\underline{Conclusions / Analysis and Discussions}}}\\\\
\Large{\textbf{\underline{References}}}\\\\
\newpage

\LARGE\section{Introduction}
\large\textbf{{Paragraph 1 (Context)}}: Area of research, current trends/ issues/ limitations, challenges to overcome etc \\\\
\large\textbf{{Paragraph 2 (Research Gap)}}: Need of the day, Research gap to fulfill the need, motivation of the work \\\\
\large\textbf{{Paragraph 3 (Scope and Objectives)}}: Scope of work and objectives, What is the objective of this work?
\begin{itemize}
    \item Objective 1: \ldots
    \item Objective 2: \ldots
    \item Objective 3: \ldots
\end{itemize}
\large\textbf{{Paragraph 4 (Proposed Approach)}}: Overview of the proposed methodology \\\\
\large\textbf{{Paragraph 5 (Contribution of the work)}}: What is the achievement in terms of new technology/concept/idea/improvement in the result etc\\\\
\large\textbf{{Paragraph 6 (Organization of the paper)}}: Organization of the paper, chapter wise statement \\\\

\hline
\section{Literature Survey}
A brief description of the related work\\
Summary Table: The existing works at a glance (~ 2 pages)\\
\hline

\section{Background}
A brief description of the background/preliminaries of the work. Briefly to the point and precisely (1-2 pages).\\
\hline

\section{Proposed Methodology}
\begin{itemize}
    \item What problems(s) have been addressed in this work?
    \item Overview of the proposed approach (try to put graphical abstracts)
    \item  Sub-section-wise detail description of the substeps involved.
\end{itemize}

Take care of the following:
\begin{itemize}
    \item What is the problem and what is its input
    \item How have you solved the problem?
    \item Why have you proposed this solution?
\end{itemize}
(4-6 pages)\\
\hline

\section{Experiment/Experimental Results/Dataset}
\begin{itemize}
    \item Detail design of experiment/setup/protocols/subjects/data collection/ or a detailed description of the used datasets
    \item Objectives of the experiment/what do you want to prove/substantiate?
    \item Experimental Environment, with a link to code/data repository
    \item Result vis-a-vis experimental objectives with proper tables, graphs, plots, etc
\end{itemize}
\hline

\section{Conclusions}
\begin{itemize}
    \item Novelty in this work
    \item Mentionable achievement/breakthrough
    \item Research/ application potential
    \item Claims if under some assumptions
    \item Future scope or further work
\end{itemize}
\hline

\section{References}
\begin{itemize}
    \item Follow a bibliography style
    \item All recent/relevant papers only
    \item 20-40 references
\end{itemize}
\hline

\section{Research Highlights}
Give 4-5 salient points of your research work, each in 10 words.












\end{document}
